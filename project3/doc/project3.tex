\documentclass[11pt]{article}
\usepackage{geometry}                % See geometry.pdf to learn the layout options. There are lots.
\geometry{a4paper}                   % ... or a4paper or a5paper or ... 
%\geometry{landscape}                % Activate for for rotated page geometry
\usepackage[parfill]{parskip}    % Activate to begin paragraphs with an empty line rather than an indent
\usepackage{graphicx}
\usepackage{amssymb}
\usepackage{epstopdf}
\usepackage{hangul}
\usepackage{color}
%\DeclareGraphicsRule{.tif}{png}{.png}{`convert #1 `dirname #1`/`basename #1 .tif`.png}

\title{Project3}
\author{Kim, Seongjun}
%\date{2010-07-22}                                           % Activate to display a given date or no date

\newcommand{\head}[1]{\LARGE \textbf{#1} \normalsize \\}

\begin{document}
\maketitle
\head{Goal}
Make B+-tree index.

The key is `int' type. Its size is 4 or 8 bytes (depends on machine). The data is pointer type and its size is also 4 or 8 byte. The block's size is 4096 bytes. If B+-tree is on 32-bit machine, leaf node has $\lfloor (4096-4) / 4 \rfloor = 1023$ keys and pointers.

Insert and remove operations are trivial. But search operation is subtle. Because this B+-tree allows duplicate keys, search operation returns a iterator that points leftmost data of duplicate key. You can access all data by using this iterator. If it can't find key, it return null iterator.

\Large \textcolor{red}{Following `BPTree.h', implement all public methods of BPTree class in `BPTree.cc'.} \normalsize

\emph{Warning: Don't change declaration of public method of BPTree class. If you need, first let me know why it is need and how to change by e-mail.}
\\

\head{How to compile \& run}
	\$ make \\
	\$ ./bptree\_test
\\

\head{References}
Chapter 10

\end{document}